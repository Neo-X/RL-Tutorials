% new commands

% common formatting commands
\newcommand{\todo}[1]{\textcolor{red}{Todo:#1}}
%\newcommand{\todo}[1]{\textcolor{red}{}}
\newcommand{\glen}[1]{\textcolor{blue}{Glen:#1}}
% \newcommand{\glen}[1]{\textcolor{blue}{}}
\newcommand{\michiel}[1]{\textcolor{green}{Michiel:#1}}
\newcommand{\todocite}[1]{\textcolor{red}{[??#1??]}}
\newcommand{\changes}[1]{{\color{green}{#1}}~}

% % % % % % % % % % % short forms % % % % % % % % % %
\newcommand{\agent}{\emph{agent}\xspace}
\newcommand{\BulletPhysics}{BulletPhysics\xspace}
\newcommand{\ANNs}{ANNs\xspace}
\newcommand{\ANN}{ANN\xspace}
\newcommand{\torso}{torso\xspace}
\newcommand{\anchor}{anchor\xspace}
\newcommand{\anchorArm}{anchor-arm\xspace}
\newcommand{\bodyArm}{body-arm\xspace}
\newcommand{\bodyMass}{body-mass\xspace}
\newcommand{\freeArm}{free-arm\xspace}
\newcommand{\anchors}{anchors\xspace}
\newcommand{\brachiator}{pendulum\xspace}
\newcommand{\brachiators}{pendulums\xspace}
\newcommand{\ricochetal}{ricochetal\xspace}
\newcommand{\flight}{flight\xspace}
\newcommand{\contractionStrength}{\emph{contractionStrength}\xspace}
\newcommand{\anchorArmStrength}{\emph{anchorArmStrength}\xspace}
\newcommand{\transitionTuples}{\emph{transition tuples}\xspace}

\newcommand{\stateFlight}{\emph{Flight}\xspace}
\newcommand{\stateSwing}{\emph{Swing}\xspace}
\newcommand{\releaseAngle}{\emph{release-angle}\xspace}
\newcommand{\armDisplacement}{\emph{arm-displacement}\xspace}
\newcommand{\flightPhaseLength}{\emph{flight-phase-length}\xspace}



% % % % % % % % % RL shortforms % % % % % % %
\newcommand{\epoch}{\emph{epoch}\xspace}
\newcommand{\policy}{\emph{policy}\xspace}
\newcommand{\eGreedy}{\ensuremath{\epsilon}-greedy\xspace}




% % % % % % % % math defines  % % % % % %
\newcommand{\parameter}{\ensuremath{p}\xspace}
\newcommand{\parameterVector}{\ensuremath{\mathbb{P}}\xspace}
\newcommand{\kernelWidth}{\ensuremath{d}\xspace}
\newcommand{\freeArmTorque}{\ensuremath{\tau_{f}}\xspace}
\newcommand{\anchorArmTorque}{\ensuremath{\tau_{a}}\xspace}
\newcommand{\anchorArmTorqueInitial}{\ensuremath{\tau_{a0}}\xspace}
\newcommand{\torsoTorque}{\ensuremath{\tau_{t}}\xspace}
\newcommand{\velocity}{\ensuremath{\textbf{v}}\xspace}
\newcommand{\position}{\ensuremath{\textbf{p}}\xspace}
\newcommand{\controllerVelocity}{\ensuremath{\velocity_{com}}\xspace}
\newcommand{\initialAngle}{\ensuremath{\theta_{0
}}\xspace}
\newcommand{\currentAnchorPosition}{\ensuremath{\position_{cur
}}\xspace}

\newcommand{\setOfTuples}{\ensuremath{\mathcal{T}}\xspace}
\newcommand{\transitionTuple}{\ensuremath{T}\xspace}

% % % % % % RL math % % % % % % %
\newcommand{\state}{\ensuremath{\textbf{s}}\xspace}
\newcommand{\nextState}{\ensuremath{\state'}\xspace}
\newcommand{\action}{\ensuremath{a}\xspace}
\newcommand{\nextAction}{\ensuremath{\action'}\xspace}
\newcommand{\reward}{\ensuremath{r}\xspace}
\newcommand{\vvalue}{\ensuremath{q}\xspace}
\newcommand{\states}{\ensuremath{S}\xspace}
\newcommand{\actions}{\ensuremath{A}\xspace}
\newcommand{\availableActions}{\ensuremath{A_{\ttime}}(\state)\xspace}
\newcommand{\ttime}{\ensuremath{t}\xspace}
\newcommand{\actionEpsilon}{\ensuremath{\epsilon}\xspace}
\newcommand{\actionOmega}{\ensuremath{\omega}\xspace}
\newcommand{\policyFunction}[2]{\ensuremath{\pi_{#2}(#1)}\xspace}
\newcommand{\valueFunction}[2]{\ensuremath{V_{#2}(#1)}\xspace}
\newcommand{\valueFunctionSampled}[2]{\ensuremath{\hat{v}_{#2}(#1)}\xspace}
\newcommand{\qValueFunction}[3]{\ensuremath{Q_{#3}(#1,#2)}\xspace}
\newcommand{\approximateQValueFunction}[3]{\ensuremath{\hat{Q}_{#3}(#1,#2)}\xspace}
\newcommand{\qValueFunctionSampled}[3]{\ensuremath{\hat{q}_{#3}(#1,#2)}\xspace}
\newcommand{\expectedValue}{\ensuremath{\mathbb{E}}\xspace}
\newcommand{\discountFactor}{\ensuremath{\gamma}\xspace}
\newcommand{\learningRate}{\ensuremath{\alpha}\xspace}
\newcommand{\rewardFunction}[3]{\ensuremath{R(#1,#2,#3)}\xspace}
\newcommand{\tranitionProbability}[3]{\ensuremath{T(#1,#2,#3)}\xspace}


% % % % % % Parametric RL % % % % % % %
\newcommand{\weightMatrix}{\ensuremath{W}\xspace}
\newcommand{\basisVector}{\ensuremath{b}\xspace}
\newcommand{\inputVector}{\ensuremath{\textbf{x}}\xspace}
\newcommand{\stocasticVariable}{\ensuremath{Y}\xspace}
\newcommand{\class}{\ensuremath{y}\xspace}
\newcommand{\predictedClass}{\ensuremath{\class_{class}}\xspace}
\newcommand{\modelParameters}{\ensuremath{\theta}\xspace}
\newcommand{\trainingData}{\ensuremath{D}\xspace}
\newcommand{\likelihood}{\ensuremath{\mathcal{L}}\xspace}
\newcommand{\loss}{\ensuremath{\ell}\xspace}
\newcommand{\qValueFunctionParamaetric}[4]{\ensuremath{q_{#4}(#1,#2,#3)}\xspace}
\newcommand{\bellmanError}[0]{\ensuremath{e_{b}}\xspace}
\newcommand{\overestimationError}[0]{\ensuremath{e_{o}}\xspace}

% math operations

\DeclareMathOperator*{\argmin}{arg\,min}
\newcommand{\specialcell}[2][c]{%
  \begin{tabular}[#1]{@{}c@{}}#2\end{tabular}}
  \DeclareMathOperator*{\argmax}{arg\,max}
  \newcommand{\specialcelll}[2][c]{%
    \begin{tabular}[#1]{@{}c@{}}#2\end{tabular}}

\newcolumntype{M}{>{\centering\arraybackslash}m{\dimexpr.68\linewidth-1\tabcolsep}}
\newcolumntype{L}{>{\centering\arraybackslash}m{\dimexpr.30\linewidth-1\tabcolsep}}